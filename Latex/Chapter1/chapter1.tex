% Activate the following line by filling in the right side. If for example the name of the root file is Main.tex, write
% "...root = Main.tex" if the chapter file is in the same directory, and "...root = ../Main.tex" if the chapter is in a subdirectory.

\chapter{Introduction}

Most undergraduate Mathematics programs require students to take 
a proof-writing course, as proofs insert formal rigor into the 
study of Mathematics. Still, there’s some degree of uncertainty, 
since human error is still involved in evaluating the proofs and 
determining their consistency. [Insert examples of theorems 
which have been erroneously proven throughout history]. For 
simpler proofs, the risk of human error is a lesser problem, 
since a discerning eye can quickly catch holes in a faulty proof. 
Nonetheless, grading the quality of a slew of proofs is 
time-consuming, especially when factoring in the time a grader 
might take to provide feedback for the specific errors in a 
poorly-written proof. The Lean Theorem prover offers a solution 
to the first problem, and this project is an effort to simplify 
the grading process of proofs by integrating Lean in a 
proof-writing course.

\section{What is Lean?}

Lean uses dependent type theory and the Curry-Howard isomorphism 
to build a programming language capable of defining and proving 
theorems in Mathematics. Although other theorem provers exist, 
Lean was chosen for this project due to its 
extensive math library, Mathlib, and its large, active community 
of users. Lean also has a “tactic mode,” wherein all of the 
premises and goals of a proof are displayed [insert screenshots] 
as they develop throughout the proof, and functions called 
“tactics” can be used to advance through the proofs using 
automation and type inference. Lean’s large user base and 
committed development team have also created a variety of 
supplemental resources for new users to learn Lean and for 
experienced users to reference, most notable of which is 
Mathematics In Lean.

\section{Why use Lean?}

\begin{enumerate}
    \item
Autograding! Any proof written in Lean which compiles is 
guaranteed to be correct.
    \item
Lean Infoview makes writing and reading proofs easier; the precise 
state of what is already proven and what remains to be shown is 
made explicit for every step of the proof.
    \item 
Tactics can automate some of the tedious parts of a proof, like 
unfurling definitions and other procedures which can be done 
algorithmically.
    \item
All of the essential parts of a proof are made explicit; 
hypotheses which go unused are automatically marked by Lean, 
and each use of some hypothesis or lemma is tracked by its 
variable name. 
\end{enumerate}
Downsides: 
\begin{enumerate}
    \item[$\cdot$]
Some parts of a proof which might be natural or trivial may be 
challenging to formalize.
    \item[$\cdot$]
Appealing to some well-known lemma requires knowledge 
of its name in Lean.
    \item[$\cdot$]
Lean proofs are not written like human language proofs, 
and don’t directly develop a student’s capacity to write 
readable human language proofs.
    \item[$\cdot$]
Lean takes some time to learn, time which could instead be used 
for purely mathematical pursuits.
    \item[$\cdot$]
Lean requires some level of tech savviness to access, 
since it must be downloaded and configured, and either the 
terminal or a code editor is necessary to write Lean code. 
\end{enumerate}
\section{Who else uses Lean?}
Heather Macbeth
[Insert information about her course]
[Insert other examples]