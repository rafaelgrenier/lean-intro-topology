% Activate the following line by filling in the right side. If for example the name of the root file is Main.tex, write
% "...root = Main.tex" if the chapter file is in the same directory, and "...root = ../Main.tex" if the chapter is in a subdirectory.

%!TEX root = 

\chapter[Development]{Development Process}

This honors thesis project began in June 2023, stemming from
an idea Dr. Sergey Cherkis had for automating the grading
process in his topology class. Over the summer, I endeavored
to learn Lean and become comfortable using Mathlib. In the Fall,
I began building the library, and I finished in the Spring.

\section{Summer 2023}

Lean has many resources available to newcomers, so I began
by tackling the introductory textbooks Mathematics In Lean and
Theorem Proving In Lean. Mathematics In Lean is peppered with
exercises for the reader, which I dutifully completed. When
I read through MIL, it was still written for Lean3, and I only 
covered the sections introducing Lean itself, set theory, and
topology.

The topology section of Mathematics in Lean was sparse, and took 
an approach to Topology centered in the notion of filters. 
\begin{definition}
    A $\mathbf{Filter}\;F$ is a collection of sets over a space $X$
    such that \begin{enumerate}
        \item if $S\in F$ and $\hat{S}$ is a superset of $S$, then
        $\hat{S}\in F$, and
        \item for any two sets $S,T\in F$, then  the intersection $S\cap T\in F$.
    \end{enumerate}
\end{definition}

% MIL's rough introduction to topology inspired me to create an 
% alternate educational resource for topology in Lean.

% I took notes on Theorem Proving in Lean

% I read some of Functional Programming

% Revisited MIL which had been updated to Lean4 in August

\section{Fall 2023}

In Late August, I began to work through the Munkres textbook
"Topological Spaces" with a focus on identifying which parts
of the textbook were more or less challenging to formalize. 
Starting in September, I pivoted to working on the first section
of my instructional repository, Logic in Lean. I spent the first few 
weeks of september deciding how to structure the introduction to 
formal logic, which ultimately culminated in the current path
starting with Propositions, truth and falsity, then leading through
implication, disjunction, conjunction, negation, and the 2 quantifiers.
I actually began writing code and comments in late september, and by
mid-october I had written files explaining proofs, proposition, and
implication; true, false, introduction rules, and elimination rules;
negation; conjunction and disjunction; and the existential and 
universal quantifiers. This was also the time when I began backing 
my files up on github. The remainder of the semester was spent 
continuing to flesh out the library, adding a section on set theory
and a section on Topology.

\section{Spring 2023}

Over the Winter break, it was confirmed by the University of Arizona
mathematics department that Dr. Cherkis would teach a graduate class
in the coming spring, Math 529: Proof Writing and Proof Checking with
a Computer. Furthermore, I was accepted as a Teaching Assistant for
this class. This class gave me the opportunity to discern how 
approachable formalization of Mathematics in Lean is in an actual
classroom environment. During this semester, I also began writing 
this Honors thesis, and I finished working on the code library
which introduces students to Lean. 

\subsection{Finishing the library}

\subsection{Insights as a TA}

For the first month and a half of the class, Dr. Cherkis lectured
in parallel to the structure of Mathematics in Lean 4. I was taken
aback by the difficulty students had in grasping the "apply" tactic,
which transforms a tactic state with goal $Q$ to a tactic state
with goal $P$ by using an implication $P\to Q$.

\subsection{Honors Thesis development}