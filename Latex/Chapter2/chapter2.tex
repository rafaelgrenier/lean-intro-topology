% Activate the following line by filling in the right side. If for example the name of the root file is Main.tex, write
% "...root = Main.tex" if the chapter file is in the same directory, and "...root = ../Main.tex" if the chapter is in a subdirectory.

%!TEX root = 

\chapter[Development]{Development Process}

This honors thesis project began in June 2023, stemming from
an idea Dr. Sergey Cherkis had for automating the grading
process in his topology class. Over the summer, I endeavored
to learn Lean and become comfortable using Mathlib. In the Fall,
I began building the library, and I finished in the Spring.

\section{Summer 2023}

Lean has many resources available to newcomers, so I began
by tackling the introductory textbooks Mathematics In Lean and
Theorem Proving In Lean. Mathematics In Lean is peppered with
exercises for the reader, which I dutifully completed. When
I read through MIL, it was still written for Lean3, and I only 
covered the sections introducing Lean itself, set theory, and
topology.

The topology section of Mathematics in Lean was sparse, and took 
an approach to Topology centered in the notion of filters. 
\begin{definition}
    A $\mathbf{Filter}\;F$ is a collection of sets over a space $X$
    such that \begin{enumerate}
        \item if $S\in F$ and $\hat{S}$ is a superset of $S$, then
        $\hat{S}\in F$, and
        \item for any two sets $S,T\in F$, then  the intersection $S\cap T\in F$.
    \end{enumerate}
\end{definition}

% MIL's rough introduction to topology inspired me to create an 
% alternate educational resource for topology in Lean.

% I took notes on Theorem Proving in Lean

% I read some of Functional Programming

% Revisited MIL which had been updated to Lean4 in August

\section{Fall 2023}

In Late August, I began to work through the Munkres textbook
"Topological Spaces" with a focus on identifying which parts
of the textbook were more or less challenging to formalize. 
Starting in September, I pivoted to working on the first section
of my instructional repository, Logic in Lean. I spent the first few 
weeks of september deciding how to structure the introduction to 
formal logic, which ultimately culminated in the current path
starting with Propositions, truth and falsity, then leading through
implication, disjunction, conjunction, negation, and the 2 quantifiers.
I actually began writing code and comments in late september, and by
mid-october I had written files explaining proofs, proposition, and
implication; true, false, introduction rules, and elimination rules;
negation; conjunction and disjunction; and the existential and 
universal quantifiers. This was also the time when I began backing 
my files up on github. The remainder of the semester was spent 
continuing to flesh out the library, adding a section on set theory
and a section on Topology.

\section{Spring 2023}

Over the Winter break, it was confirmed by the University of Arizona
mathematics department that Dr. Cherkis would teach a graduate class
in the coming spring, Math 529: Proof Writing and Proof Checking with
a Computer. Furthermore, I was accepted as a Teaching Assistant for
this class. This class gave me the opportunity to discern how 
approachable formalization of Mathematics in Lean is in an actual
classroom environment. During this semester, I also began writing 
this Honors thesis, and I finished working on the code library
which introduces students to Lean. 

\subsection{Finishing the library}

My original plan for the project
included a subsection in the Set Theory chapter of the library for
cardinality which would cover the explicit definitions of finiteness, 
countability, and uncountability, but after several weeks of work on 
the code without any forward progress, I decided to cut cardinality
from the code library. The first roadblock I encountered trying to
formalize Munkres' approach to cardinality was a paralysis of possibility, 
since Lean has finiteness defined for sets and types, each of which are
useful in select situations. Mathlib has the type \lean{Finset}, which is 
built atop Lean's programming infrastructure for lists, so all terms 
of type \lean{Finset} are constructed explicitly by enumerating their 
elements. This extensional finite set construction makes proving theorems about
cardinality all but trivial, but requires some type of external means
to interpret intensionally defined sets as \lean{Finset}s. Mathlib also 
has the type \lean{Fin : Nat $\to$ Type} and the 
predicate \lean{Finite : Type $\to$ Prop}. For \lean{n : Nat}, 
\lean{Fin n} is a subtype of \lean{Nat} consisting of all natural numbers
less than \lean{n}. \lean{Finite} is defined in terms of \lean{Fin}, where
a type is \lean{Finite} if there exists a bijective function from
that type to some \lean{Fin n}. This approach to finiteness matches the formal
approach described by Munkres, but requires introducing extraneous concepts
from the Mathlib library used for the construction of \lean{Finite}, namely
\lean{Equiv} and \lean{Subtype}. I attempted writing some files in the code
library relying on each approach, since \lean{Finset} is actually used 
later on in topology where finite subcovers and finite intersections are 
concerned, but only \lean{Finite} lends itself to the formal proofs of
the uniqueness of cardinality and so on. \lean{Finite} is also most similar
to how countability and uncountability are defined in Mathlib, so it provides
a better segue. However, I was met with another significant hurdle when trying
to prove that cardinality is unique using \lean{Finite}. Extending a function's 
domain to a larger type requires using Lean's if-then-else logic, which
is tricky to use in tactic proofs. Even if the if-condition is met by 
one of the hypotheses, some elbow grease is necessary to convince Lean to
simplify the if-then-else expression. This felt like an unreasonable onus 
to place upon students learning Lean and set theory for the first time, so 
I spent several hours searching for a workaround, but to no avail. 

Ultimately,
I decided to scrap the Countability section, for the sake of producing a 
cohesive code library before the end of the semester. Consequently, I also
needed to drop the Compactness section of the Topology chapter of
the code library. Due to time pressure, I pared down the Topology chapter
even further, leaving only the first section to be written. By April 2024, 
I had finished writing all the remaining code within the limited scope, 
including solutions for all the exercises within the code library.

\subsection{Insights as a TA}

Dr. Cherkis constructed a curriculum for the graduate level course 
which spent the first two months leading students through the first 6
chapters of \textit{Mathematics in Lean}, then took a handful of weeks to 
guide students through my code library, before returning to finish 
MIL in April. The class met twice weekly for 75 minutes, and a typical class
session spent about 20 minutes on lecture, and the remaining time dedicated
to individual or paired work on students' computers with Lean exercises. During
the group and individual work time, Dr. Cherkis and I helped students upon 
request, clarifying concepts or quirks of Lean.

\subsection{Future Development}