% Activate the following line by filling in the right side. If for example the name of the root file is Main.tex, write
% "...root = Main.tex" if the chapter file is in the same directory, and "...root = ../Main.tex" if the chapter is in a subdirectory.

%!TEX root = 

\chapter{Structure}

\section{Introduction to Lean and Logic}

Conventionally, Propositional Logic is taught with a few basic ideas:
A proposition is any statement which can be binarily assigned true or 
false, and every proposition must be either true or false. Next introduced 
are truth tables, and different means of combining propositions into 
larger propositions. We define the meaning of conjunction by appealing to 
a truth table, wherein $P\wedge Q$ is true only if $P$ is true and $Q$ is true.
Similar constructions define negation, disjunction, and implication. 

In Lean, every term has some Type, and true-or-false claims have the type 
$\mathit{Prop}$, which is short for Propositions. Any given term $P$ with
type $Prop$ is itself a type, and a term $hp : P$ is understood to be a 
proof of the proposition $P$. Therefore any function which produces as its
output a term of type $P$ is a means by which $P$ can be proven. This gives 
us a vehicle by which we can conceive of implication! If there is a function
$f$ which takes as its argument some term $hp$ of type $P$ and returns
a term of type $Q$, then $f$ is a term with type $P\to Q$. Provided that 
$P$ and $Q$ are terms with type $Prop$, $f$ can be throught of as a proof
that $P$ implies $Q$. This means of creating implication as a function from
one Proposition to another is known as the Curry-Howard Isomorphism, and
provides the basis for Lean as a proof assistant. 

The above mentioned introductions to logic are very different, so I needed
to make a choice with my code library: Do I use adapt truth tables into
Lean and teach basic logic that way, or do I introduce basic logic through
introduction and elimination rules, as is infitting with how Lean itself 
is organized? I chose to follow the path tread by the structure of Lean itself,
especially because actual mathematical proof typically abandons truth tables
before long. Therefore my code library starts with a brief explanation of
propositions and proofs, where proofs are functions from one proposition to
another. I then introduce tactics and tactic states, the other main feature
of Lean. Tactics make proof-writing in Lean flow more smoothly and resemble 
proof-writing in Human language more than pure functional code. The tactic
state also represents to the reader/writer of Lean code the names of all
hypotheses already known (the function arguments and local variables), as 
well as all the goals of the proofs (the type which the function should return).
Each new tactic employed updates the tactic state, so proving a theorem with
tactics amounts to writing tactics until the tactic state represents that there
are no goals left to be solved. I equip students with knowledge of just 3 basic
tactics in the beginning: $intro$, $apply$, and $exact$.

The $intro$ tactic functions similarly to "let" in a human language proof. 
In some proof about Natural numbers, one might write "let n be a natural number."
Similarly, in Lean, one would write $intro\;n$.
If the current goal in the tactic state is in the form 'P → Q', then
"intro hP" would update the tactic state so that there's a new hypothesis
hP : P and a simplified goal Q. "hP" is an arbitrary name here, whatever
sequence of alphanumeric characters follows the whitespace after "intro" will
be the name assigned to the term introduced. 

The $exact$ tactic is used for finishing off a proof, and might be compared
to the phrase "Because blank, the proof is complete." If the current goal is
"P" and there is some hypothesis "hp : P", then writing "exact hp" would 
complete the proof.

The $apply$ tactic uses implications to change the goal. For example, 
given the goal 'Q' and the hypothesis "h : P $\to$ Q," writing
$apply\;h$ would transform the tactic state such that the new goal
is 'P'. Since it's known that P implies Q, proving P would suffice to 
prove Q, and the apply tactic packages that logic into a single line.

The code library then works through True, False, Not, And, Or, and Iff.
True and False are both propositions, where True.intro is a function which
returns a proof of True and requires no arguments, and False.elim is a function
which takes a proof of False as an input and can output a proof of any
proposition. Not has type $Prop \to Prop$ and is denoted $\lnot$, so for
some Prop $P$, $\lnot P$ is also a Prop. $Not\;P$ is defined as
$P\to False$. $And$, $Or$, and $Iff$ are all terms of Type 
$Prop\to Prop\to Prop$, and are denoted by 
$\wedge$, $\vee$, and $\leftrightarrow$ respectively. These logical connectives
all have introduction and elimination rules; introduction rules for
creating a term with the type of the connective, and elimnation rules for
turning terms with the type of the connective into proofs of other propositions.
For example, $And.intro$ takes proofs of $P$ and $Q$ and returns a proof of
$P\wedge Q$. $And$ has two elimination rules, $And.left$ takes a proof
of $P\wedge Q$ to a proof of $P$, and $And.right$ takes a proof of 
$P\wedge Q$ to a proof of $Q$. 

\subsection{Dependent Types and Qualifiers}

Lean implements "Dependent Type Theory," which allows for the quantifiers
$\forall$ and $\exists$ to be represented in Lean. If $\alpha$ is some
type and $p$ has type $\alpha\to Prop$, \textbf{TBD}

\subsection{Are "to be" are "not to be" the only options?}

Using the the Curry-Howard Isomorphism and defining logical connectives
with introduction and elimination rules creates a system of logic which
is almost as expressive as classical logic, but it has a few shortcomings.
It's not possible to prove $P\vee\lnot P$ for an arbitrary proposition $P$
with constructive logic, the logic we have been using in Lean thus far.
Lean introduces three axioms in order to prove the law of the
excluded middle: propositional extensionality, functional
extensionality, and an axiom of choice. Propositional extensionality is the
axiom that equivalent propositions are entirely equal, functional extensionality
is an axiom stating any two functions which return the same outputs for any
given input are entirely equal, and the choice axiom is a function which produces
an term of an arbitrary type $\alpha$ given that $\alpha$ is a nonempty
type. Then using Diaconescu's theorem, it's possible to show that for any
proposition $P$, either $P$ or $\lnot P$ is true. 

The code library doesn't dwell very long on the differences between 
constructive and classical logic, nor does it discuss the measures
taken by the Lean library to extend its expressive capabilities to 
cover classical logic. The library is geared towards an audience of 
undergradute mathematics majors, so most of the time is spent developing
their understanding of classical logic.


\section{Set Theory}

The structure of the Set Theory section of the code library is modeled after 
\textit{Topological Spaces} by James Munkres and 
\textit{An Introduction to Proof through Real Analysis} by Trench, along with
some influence from the undergraduate course I took at the University of Arizona
in Formal Proof-Writing. The code library begins with a discussion of basic
set theory, then has subsections for Relations, Induction, and Cardinality. 
Induction is introduced here rather than in the opening logic section of the 
library because it requires more type theory to understand how induction is
implemented in Lean. As students work through the code library for set theory,
they will also be learning bits and pieces of type theory and how type theory
is used in Lean. 

\section{Topology}

This section of the code library is modeled more explicitly after 
\textit{Topological Spaces} by James Munkres, but also includes
some influence from \textit{Topologies and Uniformities} because 
the Lean library Mathlib uses \textit{Topologies and Uniformities}.
This final section of the code library is much less linear, with only the
basics subsection as a dependency for the other subsections. The basics
subsection introduces the idea of a topology, a topological basis, and 
a handful of methods for creating new topologies. The remainder of the 
topology section is devoted to specific topological properties:
compactness, continuity, connectedness, and separation.