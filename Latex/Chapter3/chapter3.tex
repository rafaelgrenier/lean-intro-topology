% Activate the following line by filling in the right side. If for example the name of the root file is Main.tex, write
% "...root = Main.tex" if the chapter file is in the same directory, and "...root = ../Main.tex" if the chapter is in a subdirectory.

%!TEX root = 

\chapter[Development]{Development Process}

This honors thesis project began in June 2023, stemming from
an idea Dr. Sergey Cherkis had for automating the grading
process in his topology class. Over the summer, I endeavored
to learn Lean and become comfortable using Mathlib. In the Fall,
I began building the library, and I finished in the Spring of 2024.

\section{Summer 2023}

Lean has many resources available to newcomers, so I began
by tackling the introductory textbooks \textit{Mathematics In Lean} \cite{MIL} and
\textit{Theorem Proving In Lean} \cite{TPiL}. MIL is peppered with
exercises for the reader, which I dutifully completed. When
I first read through MIL, it was still written for Lean3, and I only 
covered the sections introducing Lean itself, set theory, and
topology. When MIL was updated for Lean 4 in August, I re-read the
sections of the text which were substantially (not just syntactically)
changed from the previous version.

The topology section of Mathematics in Lean was unlike any other
introduction to topology I had ever seen, as it took 
an approach to Topology centered around filters rather than bases. 
The filter construction of topology works well for formalization 
because there are fewer cases to consider, particularly
in the realm of limits. Just looking at limits of a composition of two 
functions between metric spaces, according to MIL, there are 512 different 
cases. This plurality of slightly different cases poses little problem 
for an informal system, since one can demonstrate a particular case and
hand-wave that the other cases are similar. For formalization, 
either 512 cases must be encoded in Lean, or another system is required,
so Mathlib uses filters. The remainder of MIL's topology section
is dedicated to metric spaces and topological spaces, but both
are expressed entirely in the language of filters.

Having not been introduced to filters previously, I found the
topology section of MIL very difficult to learn from. The text seems 
to presuppose knowledge of the underlying mathematics in its readership,
and as such makes little effort to introduce the mathematical ideas, 
only taking care to delicately explain how those ideas are used in Lean.
I had also just taken a semester-long course in topology, so I was particularly 
taken aback that the only resource for topology in Lean was so unapproachable.
This became the inspirition for a new direction with my honors thesis project:
writing a code library for formalizing topology in Lean \cite{GrenierLIT} which is more accessible,
and based on a conventional approach like that of Munkres' \textit{Topology} \cite{Munkres}.

During the summer, I also read Theorem Proving in Lean 3, and then
skimmed Theorem Proving in Lean 4 when it was released later in the 
summer to catch any changes. Compared to MIL, Theorem Proving in Lean (TPiL) focuses
more on the basic constructions of Lean as a programming language, particularly
on the type theory used to formalize mathematics. My impression upon reading both
texts was that MIL expects more mathematical background knowledge from the reader
and TPiL expects more programming savviness. I also read a bit of 
\textit{Functional Programming in Lean} \cite{FPiL}, but ultimately sidelined that
textbook because it was less relevant to my goals for this project. 

\section{Fall 2023}

In Late August, I began to work through the Munkres textbook
\textit{Topological Spaces} with a focus on identifying which parts
of the textbook were more or less challenging to formalize. 
Starting in September, I pivoted to working on the first section
of my instructional repository, Logic in Lean. I spent the first few 
weeks of September deciding how to structure the introduction to 
formal logic, which ultimately culminated in the current path
starting with Propositions, truth and falsity, then leading through
implication, disjunction, conjunction, negation, and the 2 quantifiers.
I actually began writing code and comments in late September, and by
mid-October I had written files explaining proofs, proposition, and
implication; true, false, introduction rules, and elimination rules;
negation; conjunction and disjunction; and the existential and 
universal quantifiers. This was also the time when I began backing 
my files up on Github. The remainder of the semester was spent 
continuing to flesh out the library, adding a section on set theory
and a section on Topology.

\section{Spring 2024}

Over the Winter break, it was confirmed by the University of Arizona
mathematics department that Dr. Cherkis would teach a graduate class
in the coming spring, Math 529: Proof Writing and Proof Checking with
a Computer. Furthermore, I was accepted as a Teaching Assistant for
this class. This class gave me the opportunity to discern how 
approachable formalization of Mathematics in Lean is in an actual
classroom environment. During this semester, I also began writing 
this Honors thesis, and I finished working on the code library
which introduces students to Lean. 

\subsection{Finishing the library}

My original plan for the project
included a subsection in the Set Theory chapter of the library for
cardinality which would cover the explicit definitions of finiteness, 
countability, and uncountability, but after several weeks of work on 
the code without any forward progress, I decided to cut cardinality
from the code library. The first roadblock I encountered trying to
formalize Munkres' approach to cardinality was a paralysis of possibility, 
since Lean has finiteness defined for sets and types, each of which are
useful in select situations. Mathlib has the type \lean{Finset}, which is 
built atop Lean's programming infrastructure for lists, so all terms 
of type \lean{Finset} are constructed explicitly by enumerating their 
elements. This extensional finite set construction makes proving theorems about
cardinality all but trivial, but requires some type of external means
to interpret intensionally defined sets as \lean{Finset}s. Mathlib also 
has the type \lean{Fin : Nat $\to$ Type} and the 
predicate \lean{Finite : Type $\to$ Prop}. For a natural number \lean{n}, 
\lean{Fin n} is a subtype of \lean{Nat} consisting of all natural numbers
less than \lean{n}. \lean{Finite} is defined in terms of \lean{Fin}, where
a type is \lean{Finite} if there exists a bijective function from
that type to some \lean{Fin n}. This approach to finiteness matches the formal
approach described by Munkres, but requires introducing extraneous concepts
from the Mathlib library used for the construction of \lean{Finite}, namely
\lean{Equiv} and \lean{Subtype}. I attempted writing some files in the code
library relying on each approach, since \lean{Finset} is actually used 
later on in topology where finite subcovers and finite intersections are 
concerned, but only \lean{Finite} easily lends itself to the formal proofs of
the uniqueness of cardinality and so on. \lean{Finite} is also most similar
to how countability and uncountability are defined in Mathlib, so it provides
a better segue. However, I was met with another significant hurdle when trying
to prove that cardinality is unique using \lean{Finite}. Extending a function's 
domain to a larger type requires using Lean's if-then-else logic, which
is tricky to use in tactic proofs. Even if the if-condition is met by 
one of the hypotheses, some elbow grease is necessary to convince Lean to
simplify the if-then-else expression. This felt like an unreasonable onus 
to place upon students learning Lean and set theory for the first time, so 
I spent several hours searching for a workaround, but to no avail. 

Ultimately,
I decided to scrap the Countability section, for the sake of producing a 
cohesive code library before the end of the semester. Consequently, I also
needed to drop the Compactness section of the Topology chapter of
the code library. Due to time pressure, I pared down the Topology chapter
even further, leaving only the first section to be written. By April 2024, 
I had finished writing all the remaining code within this limited scope, 
including solutions for all the exercises within the code library.

\subsection{Insights as a TA}

Dr. Cherkis constructed a curriculum for the graduate level course 
which spent the first two months leading students through the first 6
chapters of \textit{Mathematics in Lean}, then took a handful of weeks to 
guide students through my code library, before returning to finish 
MIL in April. The class met twice weekly for 75 minutes, and a typical class
session spent about 20 minutes on lecture, and the remaining time dedicated
to individual or paired work on students' computers with Lean exercises. During
the group and individual work time, Dr. Cherkis and I helped students upon 
request, clarifying concepts or quirks of Lean.

Since most of my work as a TA revolved around working with students 
through the difficulties they encountered while learning Lean, the 
following few paragraphs discuss those difficulties.

As the students were beginning to learn Lean, they often stumbled with
tactics, specifically the \lean{apply} tactic, which works like so:
if you have a goal \lean{$\vdash $Q} and a hypothesis \lean{h:P$\to$Q}, 
then writing ``\lean{apply h}" will reduce the goal to \lean{$\vdash$P}. 
Students tended to forego reducing the goal and instead wrote 
``\lean{apply hP}" where they had \lean{hP:P} as a hypothesis. This common
mistake may represent that students aren't viewing known implications as
propostions themselves which require proving and specifying, just like
any other proposition. The \lean{apply} tactic also encourages a sort of
backwards proof, where the goal is continually reduced until the goal 
follows directly from the hypotheses. 

Another gripe students had with Lean was the theorem naming convention, or lack
thereof. Each theorem in the extensive Mathlib library has a unique
name, and while most theorems use the same naming convention, that
convention is not made explicit in \textit{Mathematics in Lean}. In general, 
a theorem which proves \lean{prop2} from hypothesis \lean{prop1} is named
\lean{prop2\_of\_prop1}, but the exceptions are likely more numerous than
the rule. Mathlib uses certain canonical abbreviations, so ``less than" is
always abbreviated \lean{le} and ``multiplication" is \lean{mul}. Additionally,
Lean uses dot notation for accessing fields of a structure but also
for namespace scope, so \lean{TopologicalSpace.isOpen} could be 
interpreted as a field within the structure \lean{TopologicalSpace} or as
a term defined within the namespace \lean{TopologicalSpace}. 

Fortunately, the students in this class were not the first to notice the 
difficulty of locating desired theorems and definitions, 
so there are several resources for searching through the Mathlib library. 
The website \textbf{moogle.ai}, the tactics \lean{apply?} and \lean{exact?}, 
and some other tricks can be used to ease the difficult task of finding the 
theorem you want. When writing Lean code in the VSCode editor, one can 
easily jump to the file in the library which defines a particular expression,
and this is probably true of many other code editors which support Lean. The 
tactics \lean{apply?} and \lean{exact?} perform a search through the library
for theorems which imply the goal or directly close the goal respectively. 
Lean also has keywords \lean{\#check} and \lean{\#print} which display
the type and definition respectively of whichever term follows the keyword.
Still, it took some time for students to develop an intuition for how
theorems in Mathlib are likely to be named, so an explicit guide would 
be an invaluable resource. 

Of course, this trial class wasn't all hurdles and setbacks. While the
stumbling blocks for students helped provide feedback for how to improve
my code library, there were also many aspects of the course which functioned
near seamlessly.

During the semester, Dr. Cherkis was able to set up an autograder 
through Gradescope where students could submit their Lean code proofs
for assigned theorems, and Gradescope would give them a score according to 
how many of their proofs successfully compiled in Lean. Admittedly, this 
procedure is a bit redundant, since the extra work of submitting the code
is after Lean has already successfully compiled on the student's computer.
Nonetheless, this Gradescope submission process made it possible to 
submit completed homework and receive a grade automatically, cutting down on
the work necessary for the instructor. However, there is still room for
improvement, since students needed to slightly adapt their code which compiles
locally to code which Gradscope can compile, and Dr. Cherkis needed to 
repeatedly rewrite sections of code and display files for each assignment he 
posted to Gradescope. 

By the end of the semester, the majority of the students taking the
class were self-sufficient with Lean, and could even begin to contribute
to Mathlib themselves! Students also seemed to have greater clarity 
about the mathematics which they hard worked to formalize, as was my 
experience formalizing the mathematics in this code library. 
Some students also expressed the addictive, game-like nature of formalization
in Lean which comes from the instant gratification of the compiler's feedback
when a proof is completed. During my experience with Lean, I learned about a 
lot of mathematics which was new to me, including lattice orders, filters, 
uniform spaces, and type theory. The graduate students taking this course
likely have more math experience and knowledge than I do, but I imagine 
there are some new mathematical ideas they were exposed to during the course,
aside from the greater mastery of the mathematics they already understood and
formalized.

\section{Future Development}

After the success of this graduate-level formalization course, I am
hopeful for the prospects of an undergraduate-level course which
follows my code library. There are a few other courses in other
universities which have successfully done what this project aims to do:
teach undergraduate students how to reason mathematically and write proofs
using Lean. Dr. Heather Macbeth wrote an excellent online textbook, 
\textit{The Mechanics of Proof} \cite{Macbeth}, 
which accompanies her course at Fordham University. 
Dr. Kevin Buzzard of Imperial College London also teaches a course on 
formalizing mathematics using Lean \cite{Buzzard}, 
which he has taught for a couple of 
semesters to undergraduate math students. 

For my code library to be a sufficient resource for an undergraduate course, 
it still requires some more work; the code library needs an accompanying 
textbook, human language proofs, and a grading pipeline. Ideally, the code
library would be less commented, and the content of the comments would be moved
over to a textbook, similar to how \textit{Mathematics in Lean} has both a pdf file
and a code library. Then that textbook could also contain several proofs which
are written both in Lean and in human language, so students might also
learn how to write human language proofs and not just Lean proofs. Lastly, 
my code library needs to include an easy-to-use system for creating assignments
and compiling Lean proofs for those assignments. Dr. Cherkis used Gradescope to
do this for our course, but I would like a system which requires a bit less clerical
work from both instructor and student.

My code library is also missing a subsection of Set theory which discusses
finite, infinite, countable, and uncountable sets. The library could also
reach much further into topology, covering at least continuity, compactness, and
connectedness. With the hindsight of TAing this course, the introduction section 
of the library could also be improved by taking more time on tactics, 
particularly \lean{apply} and \lean{rewrite}.

Once my code library has been upgraded enough to be used as a reference for
an undergraduate course, there are myriad possibilities for uses.
Instructors at any university could use the library for their own courses
in formal mathematics, and the library could be developed to focus less on
topology and instead anywhere else in mathematics. In my undergraduate proofs
course, topology wasn't discussed, but instead some elementary Real Analysis and
Field theory. 

I have also surmised from my experience working with Lean and helping
others to learn Lean that there \textit{really} ought to be a user guide
for tactics which explains what all the tactics do, and that Mathlib could
use some extra document which can help with searching through the library. 
The tools I have learned to find specific theorems in Mathlib were hard-fought,
but they don't have to be.

Since I began this project in June 2023, I have developed considerably
as a mathematician. Learning Lean has changed how I think about mathematics, 
as I now perceive theorems first and foremost as functions, and proofs
as implementations of those functions. This perspective clarifies how 
each premise of a theorem is necessary for arriving at the conclusions, and
makes relaxing the premises much more intuitive. All of the mathematics which
I have worked to formalize in my library has become clear to me in stark
detail, since a much clearer understanding of the underlying mathematics is
necessary to formalize and explain the formalization. I have also learned 
a fair amount of entirely new mathematics, particularly in the realm
of filters and type theory. While I intend to improve my library and see that
it finds an audience, the work I have done so far has been intrinsically
rewarding, both bolstering my skills as a mathematician and as a programmer.