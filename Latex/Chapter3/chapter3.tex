% Activate the following line by filling in the right side. If for example the name of the root file is Main.tex, write
% "...root = Main.tex" if the chapter file is in the same directory, and "...root = ../Main.tex" if the chapter is in a subdirectory.

%!TEX root = 

\chapter{Structure}

\section{Introduction to Lean and Logic}

Conventionally, Propositional Logic is taught with a few basic ideas:
A proposition is any statement which can be binarily assigned true or 
false, and every proposition must be either true or false. Next introduced 
are truth tables, and different means of combining propositions into 
larger propositions. We define the meaning of conjunction by appealing to 
a truth table, wherein \lean{P $\wedge$ Q} is true only if \lean{P} is true 
and \lean{Q} is true.
Similar constructions define negation, disjunction, and implication. 

In Lean, every term has some Type, and true-or-false claims have the type 
\lean{Prop}, which is short for Propositions. Any given term \lean{P} with
type \lean{Prop} is itself a type, and a term \lean{hp : P} is understood to be a 
proof of the proposition \lean{P}. Therefore any function which produces as its
output a term of type \lean{P} is a means by which \lean{P} can be proven. This gives 
us a vehicle by which we can conceive of implication! If there is a function
\lean{f} which takes as its argument some term \lean{hp} of type \lean{P} and returns
a term of type \lean{Q}, then \lean{f} is a term with type \lean{P $\to$ Q}. Provided that 
\lean{P} and \lean{Q} are terms with type \lean{Prop}, \lean{f} can be throught of as a proof
that \lean{P} implies \lean{Q}. This means of creating implication as a function from
one Proposition to another is known as the Curry-Howard Isomorphism, and
provides the basis for Lean as a proof assistant. 

The above mentioned introductions to logic are very different, so I needed
to make a choice with my code library: Do I use adapt truth tables into
Lean and teach basic logic that way, or do I introduce basic logic through
introduction and elimination rules, as is infitting with how Lean itself 
is organized? I chose to follow the path tread by the structure of Lean itself,
especially because actual mathematical proof typically abandons truth tables
before long. Therefore my code library starts with a brief explanation of
propositions and proofs, where proofs are functions from one proposition to
another. I then introduce tactics and tactic states, the other main feature
of Lean. Tactics make proof-writing in Lean flow more smoothly and resemble 
proof-writing in Human language more than pure functional code. The tactic
state also represents to the reader/writer of Lean code the names of all
hypotheses already known (the function arguments and local variables), as 
well as all the goals of the proofs (the type which the function should return).
Each new tactic employed updates the tactic state, so proving a theorem with
tactics amounts to writing tactics until the tactic state represents that there
are no goals left to be solved. I equip students with knowledge of just 3 basic
tactics in the beginning: \lean{intro}, \lean{apply}, and \lean{exact}.

The \lean{intro} tactic functions similarly to "let" in a human language proof. 
In some proof about Natural numbers, one might write "let n be a natural number."
Similarly, in Lean, one would write \lean{intro n}.
If the current goal in the tactic state is in the form \lean{P $\to$ Q}, then
\lean{intro hP} would update the tactic state so that there's a new hypothesis
\lean{hP : P} and a simplified goal \lean{$\vdash$ Q}. \lean{hP} is an arbitrary name here, whatever
sequence of alphanumeric characters follows the whitespace after \lean{intro} will
be the name assigned to the term introduced. 

The \lean{exact} tactic is used for finishing off a proof, and might be compared
to the phrase "Because \_\_\_\_, the proof is complete." If the current goal is
\lean{$\vdash$ P} and there is some hypothesis \lean{hp : P}, 
then writing \lean{exact hp} would complete the proof.

The \lean{apply} tactic uses implications to change the goal. For example, 
given the goal \lean{$\vdash$ Q} and the hypothesis \lean{h : P $\to$ Q}, writing
\lean{apply h} would transform the tactic state such that the new goal
is \lean{$\vdash$ P}. Since it's known that \lean{P} implies \lean{Q}, 
proving \lean{P} would suffice to prove \lean{Q}, and the \lean{apply}
tactic packages that logic into a single line.

The code library then works through \lean{True}, \lean{False}, \lean{Not}, 
\lean{And}, \lean{Or}, and \lean{Iff}.
\lean{True} and \lean{False} are both propositions, where \lean{True.intro} is a function which
returns a proof of \lean{True} and requires no arguments, and \lean{False.elim} is a function
which takes a proof of \lean{False} as an input and can output a proof of any
proposition. \lean{Not} has type \lean{Prop $\to$ Prop} and is denoted $\lnot$, so for
some \lean{P : Prop}, \lean{$\lnot$ P} is also a \lean{Prop}. \lean{Not P} is defined as
\lean{P $\to$ False}. \lean{And}, \lean{Or}, and \lean{Iff} are all terms of type 
\lean{Prop $\to$ Prop $\to$ Prop}, and are denoted by 
$\wedge$, $\vee$, and $\leftrightarrow$ respectively. These logical connectives
all have introduction and elimination rules: introduction rules for
creating a term with the type of the connective, and elimnation rules for
turning terms with the type of the connective into proofs of other propositions.
For example, \lean{And.intro} takes proofs of \lean{P} and \lean{Q} and returns a proof of
\lean{P $\wedge$ Q}. \lean{And} has two elimination rules, \lean{And.left} takes a proof
of \lean{P $\wedge$ Q} to a proof of \lean{P}, and \lean{And.right} takes a proof of 
\lean{P $\wedge$ Q} to a proof of \lean{Q}. 

\subsection{Dependent Types and Qualifiers}

Lean implements "Dependent Type Theory," which allows for the quantifiers
$\forall$ and $\exists$ to be represented in Lean. If $\alpha$ is some
type and \lean{p} has type \lean{$\alpha\;\to$ Prop}, \textbf{TBD}

\subsection{Are "to be" are "not to be" the only options?}

Using the the Curry-Howard Isomorphism and defining logical connectives
with introduction and elimination rules creates a system of logic which
is almost as expressive as classical logic, but it has a few shortcomings.
It's not possible to prove \lean{P $\vee\lnot$ P} for an arbitrary proposition \lean{P}
with constructive logic, the logic we have been using in Lean thus far.
Lean introduces three axioms in order to prove the law of the
excluded middle: propositional extensionality, functional
extensionality, and an axiom of choice. Propositional extensionality is the
axiom that equivalent propositions are entirely equal, functional extensionality
is an axiom stating any two functions which return the same outputs for any
given input are entirely equal, and the choice axiom is a function which produces
an term of an arbitrary type $\alpha$ given that $\alpha$ is a nonempty
type. Then using Diaconescu's theorem, it's possible to show that for any
proposition \lean{P}, either \lean{P} or \lean{$\lnot$P} is true. 

The code library doesn't dwell very long on the differences between 
constructive and classical logic, nor does it discuss the measures
taken by the Lean library to extend its expressive capabilities to 
cover classical logic. The library is geared towards an audience of 
undergradute mathematics majors, so most of the time is spent developing
their understanding of classical logic.

\section{Set Theory}

The structure of the Set Theory section of the code library is modeled after 
\textit{Topological Spaces} by James Munkres and 
\textit{An Introduction to Proof through Real Analysis} by Trench, along with
some influence from the undergraduate course I took at the University of Arizona
in Formal Proof-Writing. The code library begins with a discussion of basic
set theory, then has subsections for Relations, Induction, and Cardinality. 
Induction is introduced here rather than in the opening logic section of the 
library because it requires more type theory to understand how induction is
implemented in Lean. As students work through the code library for set theory,
they will also be learning bits and pieces of type theory and how type theory
is used in Lean. 

\subsection{Fundamentals of Set Theory}

The first section of the Set Theory chapter in the code library is 
a speedy introduction to Set Theory and a referral to chapter 4 of
MIL. Lean identifies sets with predicates on a type, i.e. a map
\lean{p : X $\to$ Prop} corresponds to the set \lean{S : Set X} given by
\lean{x : X | p x}. And given a set \lean{S : Set X}, the corresponding
predicate is a map \lean{p : X $\to$ Prop := $\lambda$ x $\mapsto$ x $\in$ S}. 
Since sets are
defined in terms of types, there is also a set which contains all terms 
of the given type, which Lean denotes \lean{Set.univ}
\textbf{TBD}

\subsection{Relations}

Relations are implemented in Lean as maps \lean{r : X $\to$ X $\to$ Prop}. The 
main relations discussed in this section were order and equivalence relations,
beginning with an introduction to bundling in Lean. Many objects in 
mathematics are characterized not just by the sets or functions themselves,
but also by the properties they satisfy. For example, an equivalence relation
is not just any relation on a type, but a relation which is also reflexive, 
symmetric, and transitive. Thus the object of an equivalence relation is really
4 things: the relation itself and 3 properties of that relation. Lean accomplishes
this bundling of several types into a single type with structures and typeclasses.
A structure in Lean is a type with a constructor which has all the fields
as arguments, and elimination functions which extract the individual fields.
\begin{lstlisting}
    structure <name> <parameters> <parent-structures> where
    <constructor> :: <fields>
\end{lstlisting}
For example, consider how the \lean{Equivalence} relation is expressed in Mathlib:
\begin{lstlisting}
    structure Equivalence {$\alpha$ : Type} (r : $\alpha\to\alpha\to$ Prop) : Prop where
    refl  : $\forall$ x, r x x
    symm  : $\forall$ {x y}, r x y $\to$ r y x
    trans : $\forall$ {x y z}, r x y $\to$ r y z $\to$ r x z
\end{lstlisting}
Thus the structure \lean{Equivalence} is a bundle of 3 predicates on a given
relation \lean{r}: reflexivity, symmetry, and transitivity.

Lean also has a more sophisticated system of \textbf{typeclasses}, which are
structures around which Lean has automation infrastructure. Typeclasses use 
almost the same syntax as structures, aside from the keyword \lean{class} in
place of \lean{structure}. The facet of typeclasses explained in the relations
section of the code library is \textit{typeclass inference}, which is the
mechanism Lean uses to infer instances of typeclasses without the programmer
explicitly providing the instances. Lean keeps track of the most recent scoped 
definition of each particular instance, so instances don't need to described
with specific variable names to be used. 

My section on relations investigates two main types of relations: order relations like
preorders, partial orders, well-ordering, and linear orders; and equivalence relations. Although there is 
plenty of mathematics to be done with other relations, the aim of the code library is to
provide students with a sufficient mathematical foundation to begin exploring topology.
The order topology and the quotient topology are essential to the study of basic topology, 
so I describe only the requisite information about relations for those studies. Since the order 
topology relies on intervals and rays in a linear order, the code library explores the orders
which are weaker than linear ordering but sufficient for intervals and suprema. Similarly, 
the quotient topology requires a definition of a quotient, so the code library introduces equivalence
relations and builds up to how quotients are defined in Lean.

The subsection on order relations introduces readers to the strict order, then preorders
and partial orders. Lean defines the set of upper and lower bounds of a set using only 
the preorder, but I chose to work with the confines of a partial order so that the 
bounded sets have a unique supremum and infimum. Then I introduce readers to the lexicographic
order, which will be useful later in topology. The next file on order relations delves into 
open and closed intervals and rays. Lean defines 8 different intervals (which also include rays) : 
\lean{Ioo}, \lean{Ioc}, \lean{Ioi}, \lean{Ico}, \lean{Icc}, \lean{Ici}, \lean{Iio}, and \lean{Iic}.
The 3-letter names can be thought of as acronyms where the first word is "interval", and the 
second and third words are from the set {"open", "closed", "infinite"}. The second word of the 
acronym describes the lower bound of the interval (or ray) and the third word describes the 
upper bound. The bound is a strict inequality, a non-strict inequality, or nonexistent for
"open," "closed," and "infinite" respectively. This is why there are only 8 intervals described, 
as the interval which has no upper or lower bound is just the entire set. The interval file then
introduces the reader to immediate successors and predecessors defined in terms of intervals.

The final subsection is dedicated to the equivalence relation, and is split into a file 
explaining the equivalence relation and a file explaining the quotient construction.
Lean has a structure \lean{Equivalence} which was shown above, and the code library
introduces the reader to that structure first, showing an example of a relation
which has all three properties: 
\begin{lstlisting}
    def parity (x y : $\mathbb{Z}$) : Prop := $\exists$ k, x - y = 2 * k
\end{lstlisting}
I then introduce readers to the typeclass \lean{Setoid}, which allows for the
use of the infix \lean{$\approx$} and will be needed later for quotients. The 
remainder of the file shows students how equivalence classes are defined and 
describes how equivalence classes are set partitions. Knowing now that equivalence
classes subdivide a type, the next file smoothly introduces the \lean{Quotient} type, 
where the terms are equivalence classes.

Lean actually has two quotient types, \lean{Quot} and \lean{Quotient}, where the
latter is built atop the former. \lean{Quot} is a strange kind of quotient which
is built from \textbf{any} relation, so the relation need not be an equivalence.
The traditional quotient is built in Lean as 
\lean{Quotient : Setoid $\alpha$ $\to$ Type}, so a new \lean{Quotient} type can 
be created by providing an instance of \lean{Setoid $\alpha$} for some type 
$\alpha$. The file then explains how to use this new type and how the quotient
type relates to the equivalence relation on the base type, specifically by
leading readers through a (mostly) worked example, the integers modulo 3.
I first describe a relation 
\lean{eqv (a b : $\mathbb{Z}$) : Prop := $\exists$ k, a-b = 3*k}, then
actually \textbf{prove} that \lean{eqv} is an equivalence relation. From there,
I construct an instance \lean{ZSetoid : Setoid $\mathbb{Z}$} and define the quotient
type \lean{Zmod3 := Quotient ZSetoid}. Since the terms of \lean{Zmod3} are 
equivalence classes of integers, Lean should provide a means of constructing
a term explicitly by providing an integer, and Leans does this with the function
\lean{Quotient.mk (s : Setoid $\alpha$) (a : $\alpha$) : Quotient s}. Lean also 
supplies the following useful tools for interacting with the quotient
type (and many more not listed here) :
\lean{Quotient.exists\_rep} posits that every term of the quotient type is
equal to the equivalence class of some term in the base type, 
\lean{Quotient.eq} states that two terms of the base type are equivalent if
and only if the equivalence classes of those terms are equal in the quotient type,
\lean{Quotient.lift} creates a function from the quotient type to any other type 
$\beta$ when provided some function from the base type to $\beta$ which
maps equivalent terms to the same output, and 
\lean{Quotient.lift\_mk} asserts that the function created from \lean{Quotient.lift}
maps the equivalence class to the same term that the original function would
map any representative of that equivalence class. With these theorems and some more, 
I guide readers through a proof that no perfect square of integers is two more than
any multiple of three.

\subsection{Induction}

Induction is a fundamental idea to Lean, since the language makes uses of
"inductive types." Proof by induction is also a critical idea for aspiring 
mathematicians to learn, as it's a frequently used proof technique. This
section of the code library first introduces readers to the mathematical notions
of strong and weak induction, then refers readers to \textit{Mathematics in Lean},
before returning with examples of induction and recursive definitions. Chapter 
5 section 2 of MIL was indispensible to my understanding of inductive types in 
Lean, and is accessibly-written enough for the undergraduate audience of this
code library, so I decided to send readers to that text directly rather than 
novicely rewriting the source material. 

\section{Topology}

This chapter of the code library is modeled more explicitly after 
\textit{Topological Spaces} by James Munkres, but also includes
some influence from \textit{Topologies and Uniformities} because 
the Lean library Mathlib uses \textit{Topologies and Uniformities}.
This final section of the code library is much less linear, with only the
basics subsection as a dependency for the other subsections. The basics
subsection introduces the idea of a topology, a topological basis, and 
a handful of methods for creating new topologies. The remainder of the 
topology section is devoted to specific topological properties:
compactness, continuity, connectedness, and separation.

\subsection{Topology and Bases}

